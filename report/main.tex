\documentclass[colorback,accentcolor=tud1c,11pt]{tudreport}
\usepackage[english]{babel}
\usepackage[utf8x]{inputenc}
%\usepackage[T1]{fontenc}

%\usepackage[stable]{footmisc}
%\usepackage[ngerman,pdfview=FitH,pdfstartview=FitV]{hyperref}

\usepackage{booktabs}
%\usepackage{multirow}
%\usepackage{longtable}
\usepackage{listings}
\usepackage{graphicx}
\usepackage{subfigure} 	
\usepackage{float}
\usepackage{amsmath}

\newcommand\todo[1]{\textcolor{red}{#1}}
\newcommand\code[1]{\texttt{#1}}
%\usepackage{floatflt}

\graphicspath{{./img/}}

%\newlength{\longtablewidth}
%\setlength{\longtablewidth}{0.675\linewidth}

\title{Mini-task report: Flow Map Lut-Packing Algorithm}
\subtitle{Ludwig Meysel, Mitja Stachowiak}

\begin{document}
\maketitle



\chapter{Introduction}
The task was to implement a simple version of the Flow Map algorithm which is used to bring arbitrary boolean functions and - networks to PCBs with limited size of lookup tables.


\chapter{BLIF Parser}
The Parser of the Espresso-project is used again and developed further; especially the functions are now stored in a graph-structure. It is still ready-to-use for the Espresso-package. The parser now also supports latches and sub-circuits in a basically way that enables it to hold this information in memory and save it back to file.


\chapter{Conclusion}
Niemand interessiert die Conclusion   :P
\\
The project is realized in about ???? lines of code.



\bibliographystyle{plain}
\bibliography{references}

\end{document}

